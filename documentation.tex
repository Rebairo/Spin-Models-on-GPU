\documentclass{article}

\author{Katharine Hyatt}
\title{Documentation for Lanczos and Hamiltonian generating CUDA code}

\usepackage{fullpage}

\begin{document}

\maketitle

\section{Introduction}

\section{Hamiltonian Generation}

\subsection{General Description}

There are two codes used to generate the Hamiltonians: hamiltonian.cu and hamiltonian.h. The ``builder" function, ConstructSparseMatrix, creates and fills a number and three arrays with information about the Hamiltonian for a certain model and number of sites in the COO format

\paragraph{int ConstructSparseMatrix(int, int, long*, cuDoubleComplex*, long*, long*)}

Parameters:
\begin{description}
\item[int model_Type] A value that tells ConstructSparseMatrix to build a Hamiltonian for the spin 1/2 Heisenberg (1), ..., or ... models.
\item[int lattice_Size] The number of lattice sites
\item[long* Bond] An array containing the bond information for the model we're using 
\item[cuDoubleComplex* hamil_Values] A device pointer that will have an array of Hamiltonian values allocated at it as ConstructSparseMatrix runs
\item[long* hamil_PosRow] A device pointer that will have an array of row indices allocated at it as ConstructSparseMatrix runs, in the COO form
\item[long* hamil_PosCol] A device pointer that will have an array of column indicies allocated at is as ConstructSparseMatrix runs, in the COO form
\end{description}

\section{Lanczos}

\section{References}

\end{document}
